\let\negmedspace\undefined
\let\negthickspace\undefined
\documentclass[journal]{IEEEtran}
\usepackage[a5paper, margin=10mm, onecolumn]{geometry}
\usepackage{lmodern} % Ensure lmodern is loaded for pdflatex
\usepackage{tfrupee} % Include tfrupee package

\setlength{\headheight}{1cm} % Set the height of the header box
\setlength{\headsep}{0mm}     % Set the distance between the header box and the top of the text

\usepackage{gvv-book}
\usepackage{gvv}
\usepackage{cite}
\usepackage{amsmath,amssymb,amsfonts,amsthm}
\usepackage{algorithmic}
\usepackage{graphicx}
\usepackage{textcomp}
\usepackage{xcolor}
\usepackage{txfonts}
\usepackage{listings}
\usepackage{enumitem}
\usepackage{mathtools}
\usepackage{gensymb}
\usepackage{comment}
\usepackage[breaklinks=true]{hyperref}
\usepackage{tkz-euclide} 
\usepackage{listings}                                      
\def\inputGnumericTable{}                                 
\usepackage[latin1]{inputenc}                                
\usepackage{color}                                            
\usepackage{array}                                            
\usepackage{longtable}
\usepackage{multicol}
\usepackage{calc}                                             
\usepackage{multirow}                                         
\usepackage{hhline}                                           
\usepackage{ifthen}                                           
\usepackage{lscape}
\begin{document}
	
	\bibliographystyle{IEEEtran}
	\vspace{3cm}
	
	\title{11.16.3.8.3}
	\author{EE24BTECH11047 - Niketh Prakash Achanta }
	{\let\newpage\relax\maketitle}
	
	\renewcommand{\thefigure}{\theenumi}
	\renewcommand{\thetable}{\theenumi}
	\setlength{\intextsep}{10pt} % Space between text and floats
	
	
	\numberwithin{equation}{enumi}
	\numberwithin{figure}{enumi}
	\renewcommand{\thetable}{\theenumi}
	
	
\textbf{Question}:\\
Three coins are tossed once. Find the probability of getting at least 2 heads. \\
\textbf{Solution: }\\
\textbf{Computational solution: }\\

\section*{Introduction}
This document explains the computational process of determining the probability distribution of obtaining at least 2 heads when three coins are tossed. The implementation involves two components:
\begin{itemize}
    \item A \textbf{C program} to perform the simulation, calculate the probabilities (PMF), and the cumulative distribution function (CDF).
    \item A \textbf{Python} script to use the results from the C program and generate a plot of the probability.
\end{itemize}

\section*{Definitions}
Let the random variable \( X \) represent the number of heads when three coins are tossed. The possible values of \( X \) are:
\[
X \in \{0, 1, 2, 3\}.
\]
We are interested in the probability:
\[
P(X \geq 2) = P(X = 2) + P(X = 3).
\]

\section*{C Program Implementation}
The C program performs the following steps:
\begin{enumerate}
    \item Simulate \( n \) trials of tossing three coins using the \texttt{rand()} function to generate random outcomes (head or tail).
    \item Count the occurrences of 2 and 3 heads across all trials.
    \item Compute the probability:
    \[
    P(X \geq 2) = \frac{\text{Number of occurrences of 2 or 3 heads}}{n}.
    \]
\end{enumerate}

\section*{Python Script Implementation}
The Python script:
\begin{enumerate}
    \item Loads the shared object file (\texttt{.so}) created by the C program.
    \item Calls the \texttt{calculate\_probabilities} function in the C program, passing the number of trials (\( n \)) and retrieving the PMF.
    \item Plots the PMF as a stem plot, showing the probabilities for each outcome (0, 1, 2, and 3 heads).
\end{enumerate}

\section*{Visualization}
The Python script generates a stem plot of the PMF, illustrating the probabilities for each outcome:
\[
P(X = 0), P(X = 1), P(X = 2), P(X = 3).
\]

\section*{Conclusion}
The probability of getting at least 2 heads is computed as:
\[
P(X \geq 2) = P(X = 2) + P(X = 3).
\]
This computational process combines the efficiency of C for numerical simulation and the versatility of Python for visualization.

	\begin{figure}[h!]
		\centering
		\includegraphics[width=\columnwidth]{/home/niketh/EE1003/3.8.3/figs/Figure_1.png}
		\caption{Probability distribution of the number of heads when three coins are tossed}
		\label{stemplot}
	\end{figure}
	
\end{document}

